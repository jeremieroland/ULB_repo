\message{ !name(FRIA_researchProposal.tex)}%% LyX 2.1.4 created this file.  For more info, see http://www.lyx.org/.
%% Do not edit unless you really know what you are doing.
\documentclass[12pt,british]{article}
\usepackage{mathptmx}
\renewcommand{\familydefault}{\rmdefault}
\usepackage[T1]{fontenc}
\usepackage[latin9]{inputenc}
\usepackage{geometry}
\geometry{verbose,tmargin=2cm,bmargin=2cm,lmargin=2cm,rmargin=2cm,headheight=2cm,headsep=2cm,footskip=2cm}
\usepackage{amssymb}
\usepackage{graphicx}
\usepackage{esint}

\makeatletter

%%%%%%%%%%%%%%%%%%%%%%%%%%%%%% LyX specific LaTeX commands.
%% Because html converters don't know tabularnewline
\providecommand{\tabularnewline}{\\}

\@ifundefined{date}{}{\date{}}
%%%%%%%%%%%%%%%%%%%%%%%%%%%%%% User specified LaTeX commands.

\usepackage{hyperref}
\usepackage{xcolor}
\hypersetup{
    colorlinks,
    linkcolor={red!50!black},
    citecolor={blue!50!black},
    urlcolor={blue!80!black}
}


%removes unecessary spaces from bullet listing
\usepackage{enumitem}
\setlist{nolistsep}

%for fitting more citations
\usepackage{multicol}


%to make separate references
%\usepackage[sectionbib]{chapterbib}
%\usepackage[english]{babel}
%\usepackage{biblatex}
\usepackage[style=authoryear,backend=biber,refsection=none]{biblatex}
\addbibresource{{/home/atul/SparkleShare/ULB repo/intermediate/furtherScholarship/tex/MIS2015.bib}}
\addbibresource{{/home/atul/SparkleShare/ULB repo/intermediate/furtherScholarship/tex/MIS2015}}
\addbibresource{{MIS2015}}
\addbibresource{{MIS2015.bib}}

\makeatother

\usepackage{babel}
\begin{document}

\message{ !name(FRIA_researchProposal.tex) !offset(180) }
\subsection{Research Project}

\textbf{Continuous-time communication.} The main originality of my
research project is to consider a continuous-time model to study communication
complexity and cryptographic primitives. This is in sharp contrast
with the state of the art where virtually all interactive communication
models assume discrete-time protocols, where the communication travels
back and forth between the communicating players. The continuous-time
model I propose to study assumes that the players interact via a shared
``messaging'' system that can be coupled continuously in time to their
local workspace. More precisely, assume that Alice and Bob have private
quantum registers $A$ and $B$, and share a common message register
$M$. Alice can apply a Hamiltonian $H_{A}$ to her register and an
interaction Hamiltonian $H_{AM}$ to the combined system composed
of $A$ and $M$. Similarly Bob can apply $H_{B}$ and $H_{BM}$.
The complete Hamiltonian may be written as $H=H_{A}\otimes\mathbb{I}_{MB}+H_{AM}\otimes\mathbb{I}_{B}+\mathbb{I}_{A}\otimes H_{MB}+\mathbb{I}_{AM}\otimes H_{B}.$
Note that the traditional discrete-time communication model can be
seen as a special case of this model where at any point in time, either
$H_{AM}$ or $H_{BM}$ is zero. Conversely, any continuous-time protocol
may be approached by a discrete-time protocol via a Trotter expansion.
For this reason, the continuous-time model also provides a new approach
to study the usual discrete-time model. The first task of our PhD
project will be to formalize this general framework of continuous-time
communication and its connection the usual discrete-time model, before
we can use this framework in different contexts, namely communication
complexity and the design of cryptographic primitives. 

\textbf{Communication complexity.} We can define a continuous-time
model of communication complexity by considering the time required
to evolve the system to the desired state as a measure of complexity
(for this measure to be non-trivial, we need to impose energy constraints,
in that the norms of $H_{AM}$ and $H_{BM}$ must be bounded by some
constant $\le1$). An important task will be to prove tight bounds
between the usual (discrete-time) model and this new model, using
the general connection described above. From there, we propose to
develop new techniques to bound the continuous-time communication
complexity, which will in turn imply new bounds for the usual model.
Note that the most powerful known methods to prove bounds on communication
complexity require combinatorial techniques, due to the combinatorial
structure of interactive protocols where messages are sent back and
forth. It is quite likely that the algebraic structure of continuous-time
protocols might lead to a simpler analysis and therefore bound techniques
that are easier to apply (note that there also exist algebraic bound
techniques for the usual model of communication complexity, but while
they are typically easier to apply they are also less powerful and
therefore sometimes not sufficient). Another modification of the model
we propose to exploit is an extension to quantum state generation,
where instead of computing a function $f(x,y)$ (where $x$ and $y$
are Alice\textquoteright s and Bob\textquoteright s inputs), the goal
is to create a joint quantum state $\psi_{xy}$. This extension was
quite useful to simplify the study, up to the full characterization,
of quantum query complexity, and we hope to benefit from the same
advantage here. Indeed, even if the initial and final states of a
protocol are taken from a finite set of possibilities, as is the case
when computing a function, intermediate states of the protocol can
be arbitrary, and focusing on the singular properties of the states
at the beginning and the end of the protocol might obscure the general
dynamical properties. This is especially true for the continuous-time
model I propose to explore, since in that case the intermediate states
of the protocol follow a continuous path in the Hilbert space. In
order to design new techniques to prove lower bounds on this extended
model, I propose to combine these new ideas to recent techniques for
the usual discrete-time models, in particular techniques based on
information theory, as well as bounds derived from the notion of Bell
inequalities studied in the context of quantum non-locality {[}see
above, maybe add the same references here{]}. Finally, we also intend
to study continuous time models in the purview of classical communication
complexity where the Schr�dinger equation is replaced by appropriate
Euler Lagrange equations and is expected to yield a unified characterization
of communication complexity. We intend to study similar questions;
equivalence between continuous and discrete models and characterization
of complexity models using new bounding techniques. 

\textbf{Cryptography.} Constructing explicit optimal quantum protocols
for cryptographic primitives will be among the second main targets
of this project (see Figure XXX). We will first focus on weak coin
flipping, not only because it can lead via reductions to protocols
for other primitives such as strong coin flipping and bit commitment~\cite{CK09,CK11},
but also because my promoter has already obtained some promising preliminary
results. More precisely, he has recently discovered that the best
known quantum protocol for weak coin flipping~\cite{Mochon05}, achieving
bias $1/6$, can be obtained by discretizing a continuous-time communication
protocol of the type we have defined above. This construction gives
more insight on how the protocol works and therefore provides a new
approach to extend it and break the limit of bias $1/6$, which resisted
all researchers for more than a decade. Indeed, while Mochon's original
construction requires taking the limit of an infinite number of discrete
steps, this new construction immediately yields a protocol as a continuous
evolution, which is easier to analyze and provides more intuition.
Our strategy will be to exploit this new intuition to finally break
the barrier corresponding to bias $1/6$. In order to do so, we propose
to adapt to this framework a technique introduced by Feynman in his
early works on quantum computers~\cite{Feynman86}, consisting in
constructing the history state of a computation, that is, $\left|\Psi\right\rangle =\sum_{t=0}^{T}\left|\psi_{t}\right\rangle \left|t\right\rangle $.
We believe that this might be the key to obtaining better biases,
because this states involves at the same ``physical time'' all states
of the protocol at different ``logical'' times, and therefore has
a structure which is very similar to the so-called ``time-independent
point games'' that Mochon used to prove (in a non-constructive manner)
the existence of protocols with arbitrarily low biases~\cite{Mochon07}.
In the case of a continuous-time protocol, the history state reads
$\left|\Psi\right\rangle =\int_{t=0}^{T}\left|\psi(t)\right\rangle \left|t\right\rangle dt$,
where the clock register $\left|t\right\rangle $ holds a continuous
variable, and is therefore infinite-dimensional. The fact that the
time-independent point games for bias less than $1/6$ might actually
correspond to a continuous-time history state, which would imply that
an infinite-dimensional register is necessary to translate with no
loss the point games into explicit protocols, could explain all the
failed attempts at obtaining such protocols. Once a first protocol
with bias less than $1/6$ is obtained, the obvious next step will
be to decrease the bias further, up to an arbitrary low bias, that
is, an optimal protocol. In order to do so, our main strategy will
be to iterate the process allowing to break the initial barrier, combining
history states, continuous-time evolution and continuous variables.
Next, we will turn our attention to other cryptographic primitives
such as strong coin flipping and bit commitment. It is known that
an optimal quantum protocol for weak coin flipping immediately yields
by reduction optimal protocols for these other primitives as well~\cite{CK09,CK11},
but it is quite likely that the reduction leads to unnecessary overhead
in terms of time and space. Our goal will therefore be to make these
protocols as efficient as possible by simplifying the protocols obtained
by reduction, or by directly constructing optimal protocols for these
primitives using similar techniques as for weak coin flipping. From
optimal continuous-time quantum protocols for these cryptographic
primitives, the next final step will be to derive explicit discrete-time
protocols. Here we will use the general framework we developed to
study continuous-time quantum communication protocols and their connection
with discrete-time protocols. Again, an important challenge will be
to reduce the overhead in terms of time and space. However, we also
plan to use optical quadratures to implement continuous variable registers,
an overarching goal being to obtain protocols that only involve operations
that could be realistically implemented. Specifically, a decomposition
into Gaussian operations would be preferred to enable easy experimental
implementation. The effect of imperfections must also be modelled
to quantify loss of security and robustness thereof. 

Finally, we also intend to explore the cryptographic variant of quantum
communication complexity, where the players wish to reveal as little
information as possible about their inputs $(x,y)$ while computing
$f(x,y)$. This can be achieved by first considering an \textquoteleft honest-but-curious\textquoteright{}
model which is classically equivalent to information complexity. The
quantum version is, however, not as clear where multiple definitions
of quantum information complexity coexist but they do not necessarily
yield the actual information achievable by the players. We intend
to arrive at a more suitable definition using new techniques based
on a continuous-time model. The malicious case can be subsequently
considered with the history state being harnessed to detect cheating.




\message{ !name(FRIA_researchProposal.tex) !offset(331) }

\end{document}
